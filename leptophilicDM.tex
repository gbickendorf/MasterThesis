\chapter{Leptophilic Dark Matter}
\label{ch:LepDM}
A particularly elegant way of avoiding direct detection bounds or explain why nothing has been found yet are leptophilic dark matter models where either the dark matter particles couple directly to the standard model leptons, or via another particle that mediates the interactions between dark matter and leptons. This completely avoids tree level interactions with gauge bosons and baryons that would lead to not-so-dark matter.

One way of incorporating these ideas is a hidden sector that contains only singlets under the standard model gauge group. Although this model on its own might explain the null results of direct detection efforts, this also paints the theoretically most boring picture. To connect the hidden sector to the standard model in a controlled manner, one may introduce additional fields, that mediate interactions between both sectors. These are then fittingly called portals. 

\section{Vector mediator}
The extension of the standard model by a new vector boson is well motivates both from a bottom up as well as from a top down perspective from grand unified theories. To this end the gauge group is extended by another $U_D(1)$ gauge group that is spontaneously broken such that the associated particles become massive. Such a particle is often labelled dark photon denoted by $A'$. This mediator couples additionally to the dark matter particles $\chi$. Since the exact nature of $\chi$ does not matter to this treatment it will be assumed to be a Dirac fermion for simplicity's sake.
The resulting Lagrangian can the be written as 
\begin{align*}
\mathcal{L}=& \mathcal{L}_\text{SM}-\frac{1}{4}F_{\mu\nu}'F'^{\mu\nu}+\frac{m_{A'}^2}{2}A_\mu'A'^\mu +\frac{\epsilon}{2}F_{\mu\nu}'F^{\mu\nu}-\sum_{l=e,\mu,\tau} e'_l\left(\bar{l}\gamma^\mu A'_\mu l+\bar{\nu}_l\gamma^\mu A'_\mu \nu_l\right)\\&+\bar{\chi}(i\slashed{\partial}-m_\chi)\chi -g_D\bar{\chi}\gamma^\mu A_\mu'\chi
\end{align*}
with the coupling constants $g_D$ and $e_l'$ left to be constrained and the kinetic mixing parameter $\epsilon$.
To avoid anomalies, one may chose to gauge any combination $X=yB-\sum x_iL_i$ of baryon number B and lepton family number $L_i$ that satisfy $3y=x_e+x_\mu+x_\tau$ \cite{Altmannshofer:2014pba}. Popular choices include gauged $B-L$ or $L_\mu-L_\tau$. While the former is highly constrained by direct collider searches and other 5th force experiments, the latter still exhibits rather weak constraints with a region that is even favoured by the $g_\mu -2$ anomaly.


\begin{figure}[H]
\centering
\begin{tikzpicture}
\begin{feynman}[layered layout]
\vertex (in){\(\gamma\)};
\vertex [right=of in](a);
\vertex [right=of a](b);
\vertex [right=of b](out){\(A'\)};
\diagram* {
(in)--[boson](a) -- [fermion,half left,edge label=\(f\)] (b) -- [fermion,half left] (a),
(b)--[boson](out)};
\end{feynman}
\end{tikzpicture}
\caption{Kinetic mixing of the SM-photon and the dark photon}
\label{fg:KinMix}
\end{figure}
The term proportional to $\epsilon$ has been added  to work with the most general renormalisable gauge invariant Lagrangian. 
Even though this term might be absent at tree level, as is the case for some GUT theories, it can be generated by loop contributions such as the diagram shown in figure \ref{fg:KinMix}, where the fermion running in the loop is charged both under the standard model and dark $U(1)$-group.
This leads to \cite{Rizzo:2018vlb}
\begin{equation}
\epsilon =\sum_l \frac{ee_l'}{12\pi^2}\ln\left(\frac{m_l^2}{\mu^2}\right)
\label{eq:KinMix}
\end{equation}
 and another effective interaction after diagonalization of the form 
\begin{equation}
\mathcal{L}\supset e\epsilon A'_\mu J_{em}^\mu
\end{equation}
such that now every electrically charged particle interacts with the dark photon as it is now millicharged under $U(1)_D$. This is used to put strong constraints on the parameter space.
This has also been used to explain the $g_\mu-2$ anomaly mentioned before. The discrepancy is removed with $m_{A'}<100\text{MeV}$ and $\epsilon\sim 10^{-3}$ for a kinetically mixed dark photon. But the last window of allowed parameter space has been closed by the NA48/2 Collaboration \cite{Goudzovski:2014rwa} such that there is no both problems cant be explained at once. 

If the A' is the lightest dark sector particle ($m_{A'}<2m_\chi$) it will mostly decay to the leptons $l$ it couples directly to with a the decay width
\begin{equation}
\Gamma(A' \rightarrow l^-l^+)=\frac{e_l^{'2}}{12\pi}m_{A'}\left(1+\frac{2m_l^2}{m_{A'}^2}\right)\sqrt{1-\frac{4m_l^2}{m_{A'}^2}}.
\end{equation}
This opens the scenario up to displaced vertex searches and other direct detection experiments. 

Once the decay to a $\chi$ pair is kinematically allowed ($m_{A'}>2m_\chi$) and $g_D > e'_l$ the mediator decays mainly invisibly with width 
\begin{equation}
\Gamma(A' \rightarrow \bar{\chi}\chi)=\frac{e_D^{2}}{12\pi}m_{A'}\left(1+\frac{2m_\chi^2}{m_{A'}^2}\right)\sqrt{1-\frac{4m_\chi^2}{m_{A'}^2}}
\end{equation}
such that direct detection is more difficult.  
When the dark photon only couples via kinetic mixing to the standard model the observed abundance is generated with \cite{Izaguirre:2014bca}
\begin{equation}
\alpha_D \equiv \frac{g_D^2}{4\pi} \approx 1.3\cdot 10^{-10}\left(\frac{m_{A'}}{10\text{MeV}}\right)^4\left(\frac{\text{MeV}}{m_\chi}\right)^2\frac{1}{\epsilon^2}.
\end{equation}
If $\epsilon$ is larger than this condition, $\Omega_\chi < \Omega_{\text{DM}}$ holds and $\chi$ might at most be a subdominant part of the total dark matter sector. 

Additional motivation for direct interactions with the muon specifically come from the $(g-2)_\mu$ discrepancy. The additional contribution from the vector is \cite{Kahn:2018cqs}:
\begin{equation}
\Delta a_\mu^V=\frac{e_\mu'^2}{4\pi^2}\int_0^1 d z \frac{m_\mu^2z(1-z)^2}{m_\mu^2(1-z)^2+m_{A'}^2z}
\end{equation}
which simplifies for $m_{A'}\ll m_\mu$ to
\begin{equation}
\Delta a_\mu^V \approx 1.6\times 10^{-9}\left(\frac{e_\mu'}{10^{-4}}\right)^2.
\end{equation}

\section{Scalar mediator}
Besides the vector mediator, a scalar might mediate interactions between the dark matter and the standard model. Besides the use as dark matter mediators, these models are proposed to solve the $(g-2)_\mu$ anomaly and the proton radius puzzle, depending on the coupled fermions. 
Here we propose a real scalar field $\phi$ with mass $m_\phi$ that is a singlet under the standard model gauge group and the same dark matter fermion $\chi$. The corresponding Lagrangian is 
\begin{align*}
\mathcal{L}=& \mathcal{L}_\text{SM}+\frac{1}{2}\partial_\mu \phi \partial^\mu \phi-\frac{m_{\phi}^2}{2}\phi^2 -\sum_{l=e,\mu,\tau} e'_l\bar{l} l \phi \\&+\bar{\chi}(i\slashed{\partial}-m_\chi)\chi -g_D\bar{\chi}\chi\phi.
\end{align*}
Obviously the introduction of the interaction term explicitly breaks the gauge invariance.
This might originate from a dimension 5 operator like 
\begin{equation}
\frac{c_l}{\Lambda}\phi \bar{L}_i\Phi e_{iR}+h.c.
\end{equation} 
where $\Phi$ is the standard model Higgs field that leads after SSB to the direct coupling of the scalar to the leptons with 
\begin{equation}
e'_l=\frac{c_l v}{\Lambda\sqrt{2}}.
\end{equation}
Most lepton-specific scalar mediator models considered in the literature propose the effective scalar couplings $c_l$ to be proportional to the Yukawa-coupling so that these follow the lepton mass hierarchy. 
A UV-completion above the scale $\Lambda$ will not be considered here but can be found for example as a Leptonic Higgs Portal in Ref\cite{Batell:2016ove}. Consequences besides the resulting low energy couplings are ignored for now.
This interaction leads  at one loop to an effective coupling to two photons of the form 
\begin{align*}
\frac{1}{4}g_{\gamma\gamma} \phi F_{\mu\nu} F^{\mu\nu}\\
\end{align*}
through the diagram shown in figure \ref{fg:PhotonCoupling}. The effective coupling strength $g_{\gamma\gamma}$ can be found in Ref. \cite{Chen:2018vkr}. This can be used to search for $\phi$ in the diphoton invariant mass distribution. 
The decay width to photons is
\begin{equation}
\Gamma(\phi\rightarrow \gamma \gamma)= \frac{\alpha^2m_\phi^3}{256\pi^3}\left\lvert\sum_{l=e,\mu,\tau} \frac{e'_l}{m_l}F_{1/2}(x_l,0) \right\lvert^2
\end{equation}
where $x_l = \frac{4m_l^2}{m_\phi^2}$ and and the loop function $F_{1/2}$ reads
\begin{equation}
F_{1/2}(x_l,0)= \begin{cases}
-2x_l\left[1+(1-x_l)\arcsin^2(x_l^{-1/2})\right]&x_l \geq 1\\
-2x_l\left[1-\frac{1-x_l}{4}\left(-i\pi +\log\frac{1+\sqrt{1-x_l}}{1-\sqrt{1-x_l}}\right)^2\right]&x_l<1
\end{cases}
\end{equation}

\begin{figure}[H]
\centering
\begin{tikzpicture}
\begin{feynman}[layered layout]
\vertex (in){\(\phi\)};
\vertex [right=of in](a);
\vertex [above right=of a](b);
\vertex [below right=of a](c);
\vertex [right=of b](f1){\(\gamma\)};
\vertex [right=of c](f2){\(\gamma\)};
\diagram* {
(in)--[scalar](a);
(a)--[fermion,edge label=\(l\)] (b) -- [fermion] (c) -- [fermion] (a);
(b)--[boson](f1);
(c)--[boson](f2)};
\end{feynman}
\end{tikzpicture}
\caption{One loop contribution to the $\phi$-photon coupling}
\label{fg:PhotonCoupling}
\end{figure}

For $g_D > e_l'$ and $2m_\chi < m_\phi$ the scalar mediator decays mostly invisibly to the dark sector.
The biggest decay width is then 
\begin{equation}
\Gamma(\phi \rightarrow \bar{\chi}\chi)=g_D^2\frac{m_\phi}{8\pi}\left(1-\frac{4m_\chi^2}{m_\phi}\right)^{3/2} .
\end{equation}
The main chance for collider experiments to constrain this model is then missing energy searches.
For $2m_\chi > m_\phi$ the mediator decays mostly visibly to the leptons it directly couples to as long as kinematically allowed.
\begin{equation}
\Gamma(\phi \rightarrow \bar{l}l)=e_l'^2\frac{m_\phi}{8\pi}\left(1-\frac{4m_l^2}{m_\phi}\right)^{3/2}
\end{equation}
This case behaves in collider experiments similar to the former if the scalar is long lived. If its short lived and decays at least in the majority of events in the detector material, it becomes accessible to displaced vertex or dilepton invariant mass searches.

This model has also been used to explain the $(g-2)_\mu$ results. This time its influence is slightly different:
\begin{equation}
\Delta a_\mu^S=\frac{e_\mu'^2}{16\pi^2}\int_0^1 d z \frac{m_\mu^2(1-z)(1-z^2)}{m_\mu^2(1-z)^2+m_{\phi}^2z}
\end{equation}
which simplifies for $m_{\phi}\ll m_\mu$ to
\begin{equation}
\Delta a_\mu^S \approx 6.0\times 10^{-10}\left(\frac{e_\mu'}{10^{-4}}\right)^2 .
\end{equation}