\chapter{Definitions}
Feynman slash notation:
\begin{equation*}
\slashed{\partial}=\gamma_\mu\partial^\mu
\end{equation*}
Pauli matrices:
\begin{align*}
\sigma_1=\begin{pmatrix}
0&1\\1&0
\end{pmatrix}&&
\sigma_2=\begin{pmatrix}
0&-i\\i&0
\end{pmatrix}&&
\sigma_3=\begin{pmatrix}
1&0\\0&-1
\end{pmatrix}
\end{align*}
The gamma-matrices satisfy the Clifford algebra $\{\gamma^\mu,\gamma^\nu\}=2g^{\mu\nu}$
and are represented in the Weyl basis as
\begin{align*}
\gamma^0=\begin{pmatrix}
0&I_2\\I_2&0
\end{pmatrix}&&
\gamma^i=\begin{pmatrix}
0&\sigma_i\\-\sigma_i&0
\end{pmatrix}
&&
\gamma^5=i\gamma^0\gamma^1\gamma^2\gamma^3=\begin{pmatrix}
-I_2&0\\0&I_2
\end{pmatrix}
\end{align*}
and can be used to form the chiral projection operators:
\begin{align*}
P_L=\frac{1-\gamma^5}{2}&&P_R=\frac{1+\gamma^5}{2}
\end{align*}
\paragraph{Constants}
hier noch die ganzen verwendeten konstanten hin