\chapter{Conclusion}
\label{ch:conclusion}

Here we have investigated models, that are not explicitly introduced to solve existing problems besides a possible explanation for dark matter. Nevertheless, some may at least be a subcomponent of these. The models here were for a scalar or vector mediator with masses between 0.1 to $\SI{100}{\mega \eV}$ and direct couplings only to leptons. That way these \emph{leptophilic} mediators don't leave strong signatures at high intensity proton accelerators and explain their null results for the search for dark matter. In the dark sector these mediators couple directly to the dark matter candidates $\chi$. If this coupling is large enough and the mediator decay to a $\bar{\chi}\chi$ pair is allowed, the main signatures in colliders is the missing energy which is somewhat more subtle than for example displaced vertices for visibly decaying mediators.

This thesis explored two novel ways of deriving limits on the parameter space.The first method introduced here, used the well known muon decay spectrum. The standard model decay spectrum is parametrised by Michel parameters to reflect potential additional interactions that might mediate the $\mu^+\rightarrow e^+ \nu_e \bar{\nu}_\mu$ decay besides the V-A interaction. These have been experimentally determined to great precision and yielded no significant deviation from the standard model. The main point made here, is that the only detected decay product is the positron, while the neutrinos escape detection. So processes with an additional particle coupled to any of the leptons, that also escapes detection, will be indistinguishable although it distorts the positron spectrum. This deformation was used to derive bounds on the coupling to the mediators. 

Scalar and vector mediators directly coupling to the electron result in bounds from the muon decay, that are weaker by more than one order of magnitude than missing energy searches by BABAR. Limits on the kinetic mixing parameter from BABAR or NA64 can be translated to bounds on the direct coupling between a vector mediator and the muon. Since this mixing only occurs at one loop, this translated bound is rather weak, even weaker than the derived bounds in this work. Additionally, bounds from neutrino trident production by CCFR are applicable for the vector to muon coupling model as well. The derived bounds from the muon decay are more than one order of magnitude weaker then the existing ones. Better bounds were only obtained for the scalar to muon coupling. Masses between $1$ and $\SI{20}{\mega \eV}$ and coupling constants between 0.04 and 0.2 were excluded by the muon decay, that were previously not excluded by existing experiments assuming mass hierarchical couplings $e_e'/e_\mu'=m_e/m_\mu$. The last model considered here was with gauged $L_{e-\mu}$ symmetry. Here the direct coupling to electrons result in strong existing constraints so that no new parameter space could be ruled out and the resulting bounds are weaker by more than one order of magnitude than the BABAR bounds.

Further constraints were derived from the standard model pion decay mode $\pi^+\rightarrow e^+ \nu_e$. As a two body decay the positron spectrum is a sharp peak that is experimentally broadened. The PIENU collaboration searched for additional peaks besides the standard model as a sign for heavy neutrinos. After subtracting the known standard model background, this can also be used to find bounds on the leptophilic mediators. Using a $\chi^2$ test to set 90\% and 95\% C.L. bounds, neither the vector nor the scalar coupling constraints to the electron could improve the known bounds from BABAR although the scalar results were parametrically around one order of magnitude stronger because this coupling removes the helicity suppression of this mode. Here an improvement of the experimental data by around one order might result in competitive bounds. 

Lastly the removal of the helicity suppression was used on its own to derive supplemental constraints on the scalar to electron coupling. Here a violation of apparent lepton universality would be enhanced by the scalar. Comparing the standard model prediction and experimental value of the test in pion decays sets constraints on the parameter space. Again the parameter space is already ruled out by BABAR. 

Future theoretical and experimental work continue to explore the parameter space of these models. Supplemental probes, like the ones presented above for the coupling between the scalar and muon or the scalar and electron after an improvement of the experimental sensitivity will continue to provide valuable insights. Especially since further improvement in the pion decay spectrum is well motivated by searches for heavy neutrinos, this method might become competitive. Any coupling to electrons can be investigated additionally in $e^+e^-$ colliders, since Belle II is currently running.
Couplings to the muon will continue to attract experimental effort since the $(g-2)_\mu$ anomaly is another problem waiting to be solved. If approved, the NA64-$\mu$ run or the $M^3$ experiment will reach unprecedented sensitivity to these scenarios.
All in all, the future will show exciting insights in the models considered above.