\chapter*{Abstract}
\addcontentsline{toc}{section}{Abstract}
Since its proposition, the only evidence for the existence of dark matter is still its gravitational interaction. Most models that have been proposed to solve multiple open problems for modern physics such as the axion are put under a lot of pressure. Now it is the time to consider a wider choice of models and try to constrain their parameter space to either rule them out, or find them to be a likely model for dark matter. Here we consider a secluded dark sector that exclusively interacts with some or all standard model leptons via a scalar or vector mediator and hence is referred to as 'leptophilic'. These additional interactions will be used to constrain the parameter space in two scenarios that do not need additional experimental data. 

The first interesting scenario is the standard model muon decay $\mu^+\rightarrow e^+\nu_e \bar{\nu}_\mu$. This has already been parametrised and experimentally tested as it is sensitive to non V-A interactions. If an on-shell mediator is in the final state the spectrum will be deformed and deviate from the standard model prediction. This scenario is used to derive constraints to the vector and scalar coupling to the electron and muon. Additionally a gauged $L_{e-\mu}$ model is also tested.

The second scenario is the rare $\pi^+\rightarrow e^+ \nu_e$ decay. The positron spectrum will again be altered if the leptophilic mediator couples to the electron. This is then used to derive constraints on the models. The scalar results in comparatively strong bounds, because it removes the chiral suppression of the decay to a positron. The decay width will then contribute to apparent deviations from lepton universality in pion decays that in turn can be used to derive constraints on the model.