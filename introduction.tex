\chapter{Introduction}
The understanding of the nature of dark matter (DM) is one of the still open questions of modern physics. Evidence for the existence of additional mass contributions to the universe besides the known standard model of particle physics has been piling up for decades. From the CMB at the largest scale, over the structure of galaxy clusters down to the rotation of a galaxy on its own besides gravitational lensing on intermediate scales are all essentially gravitational probes of the dark matter, leaving the exact structure obscure. Despite understanding a wealth of phenomena described by the standard model of particle physics, that at least gives the impression to understand most of physics, most of the mass of the Universe is not accounted for by known phenomena. The quest of understanding the nature and behaviour of this big and important component of our Universe concerns a wide range of branches of physics, from the study of the history and evolution of our universe by cosmologists and astronomers down to high-energy particle physicists searching for direct and indirect evidence.
To tackle this gap in the understanding of our Universe, a wide range of models have been proposed. Some dark matter candidates come almost as a side product of solutions of other problems. One of the most prominent might be the axion that resolves the strong CP-problem in the Peccei-Quinn theory. Another class of models for dark matter comes from supersymmetric models, that elegantly solves the hierarchy problem. In the minimal supersymmetric extension of the standard model the fermionic partners of the photon, the Z boson and both neutral Higgs mix to form neutralinos, which has been a widely studied candidate for dark matter.

This thesis takes the opposite path. Here two classes of models will be introduced primarily as possible mechanisms to connect the dark sector with the standard model of particle physics. The first class consists of a gauged lepton-family number, that after spontaneous symmetry breaking introduces a massive vector mediator, the dark photon, that couples to some or all leptons of the standard model and in the dark sector to Dirac dark matter. This lepton coupling introduces a kinetic mixing term with the photon and results in a small coupling between the dark photon and all electrically charged particles. 
The second class will be a scalar mediator, that also only couples directly to some or all charged leptons. Since this breaks the SU(2) symmetry, this model can at most be an effective theory.
The  Although the exact nature of the dark matter itself will stay unknown, at least this interaction might be experimentally tested.  Some of these mediators might also help solve other outstanding problems. For example a vector or scalar coupling to the muon might help elevate the $(g-2)_\mu$ problem where the anomalous magnetic moment of the muon deviates significantly from the standard model prediction. 

There is already a rich background of terrestrial experimental tests for any direct sign of these mediators. These cover a wide mass range from precision measurements of atomic spectra to searches for missing momentum or displaces vertices in high energy colliders. Nevertheless no new signs have been found up to today, so these result only in additional limits on the parameter space. This thesis aims to supplement these with an analysis of the consequences of the additional coupling to existing experimental data.

The standard model $\mu^+\rightarrow e^+ \nu_e \bar{\nu}_\mu$ decay is a well established test for the weak interaction. The positron energy- and angle-spectrum is described with Michel-parameters, that would show deviations from the standard model predictions, if another interaction mediated the decay. These parameters have been precisely measured and showed no significant difference. If a leptophilic mediator couples to the involved leptons, this should show in a deformation of this spectrum. This possible change will then be reflected by altered Michel-parameters that can then be used to derive constraints on the mediator-parameter-space.

The pion $\pi^+\rightarrow e^+ \nu_e$ decay spectrum has been measured in the past to search for heavy neutrinos. Experimentally the standard model background is well known and can be subtracted to expose any additional peaks that would indicate heavy neutrinos. Even though the introduction of the leptophilic mediators does not result in the clear peaks from the two body decay, it might still result in observable deformations. This additional component will be compared to the background and be used to derive additional constraints on the mediator-electron coupling. 

The pion decay has also been used as a test for the lepton universality of the weak interaction by comparing the expected ratio of decay-widths to muons to that to electrons. An additional coupling to the electron by the leptophilic mediators would then produce an additional and undistinguishable process that appears as another contribution to the decay width to electrons. This on its own is especially sensitive to a scalar mediator and will in turn be translated to bounds on the electron-scalar-mediator coupling.

This thesis is organised as follows:
Chapter \ref{ch:sm} introduces the standard model of particle physics and a small selections of signs that it doesn't paint the whole picture. One of its mayor shortcomings and the overlying topic, dark matter is introduced in chapter \ref{ch:DM}. The leptophilic mediator models, considered here, and some of their properties will be covered in chapter \ref{ch:LepDM}. Existing experimental constraints that will be used to benchmark the results are described in chapter \ref{ch:ExConst}. Chapter \ref{ch:SMMuon} will cover the standard model description of the muon decay. Deviations of the described spectrum will be used in chapter \ref{ch:MuBounds} to derive constraints on the parameter space of the leptophilic mediators. The standard model pion decay is described in chapter \ref{ch:SM-PionDecay}. The influence of the mediators coupling additionally to the electron on the positron energy spectrum is used in chapter \ref{ch:BoundsPI} to derive bounds on the couplings. The removal of the chiral suppression of the $\pi^+\rightarrow e^+ \nu_e$ decay is finally used on its own in chapter \ref{ch:chiSupp} to derive additional constraints. Chapter \ref{ch:conclusion} then finishes with concluding remarks.